\documentclass{article}[]

\usepackage{polski}
\usepackage[utf8]{inputenc}
\usepackage{times}
\usepackage{graphicx}
\usepackage{tabularx}
\usepackage{latexsym}
\usepackage{amssymb}
\usepackage{amsthm}
\usepackage{amsmath}
\usepackage{multirow}
%\usepackage{theorem}

%%%%%%%%%%%%%%%%%%%%%%%%%%%%%%% MACRO

\renewcommand{\S}{\mathcal{S}}
\newcommand{\pay}{\mathit{pay}}
\newcommand{\N}{\mathcal{N}}
\newcommand{\s}{\mathbf{s}}
\renewcommand{\b}{\mathbf{b}}
\renewcommand{\r}{\mathbf{r}}
\newcommand{\w}{\mathbf{w}}
\renewcommand{\P}{\mathcal{P}}

\newcommand{\G}{\mathcal{G}}
\newcommand{\U}{\mathcal{U}}
\newcommand{\tuple}[1]{\langle#1\rangle}
\newcommand{\todo}[1]{\noindent{{\small\bf$<$\textsf{#1}$>$}}}

\renewcommand{\sp}{\mathsf{sp}}
\newcommand{\mcs}{\mathsf{cs}}
\newcommand{\ssp}{\mathsf{ssp}}
\newcommand{\SC}{\mathsf{SC}}
\newcommand{\NE}{\mathsf{NE}}
\newcommand{\SE}{\mathsf{SE}}

%ALGORITMO
\newcounter{stepcount}
\newcommand{\step}{%
        \stepcounter{stepcount}%
        \thestepcount}
\newcommand{\resetstep}{%
        \setcounter{stepcount}{0}
}

\newtheoremstyle{break}
  {\topsep}{\topsep}%
  {\itshape}{}%
  {\bfseries}{}%
  {\newline}{}%
\theoremstyle{definition}
\newtheorem{zad}{Zadanie}


%\newtheorem{example}{Example}[section]
%\newtheorem{definition}{Definition}[section]
%\newtheorem{lemma}{Lemma}[section]
%\newtheorem{theorem}{Theorem}[section]

%\newenvironment{proof}{\vspace{-1mm}{\em Proof.}}{\vspace{2mm}}

\newcolumntype{Y}{>{\centering\arraybackslash}X}

\begin{document}
\title{Badania operacyjne i systemy wspomagania decyzji
\\ \Large 05 Programowanie liniowe - analiza wrażliwości\\ i postać dualna}
\date{}

\maketitle

\begin{zad}[5 pt.]
Zakład produkuje 3 rodzaje akumulatorów do samochodów: model SS (super), model S (standardowy), i model O (oszczędny). Każdy z trzech typów podlega obróbce na 3 maszynach: Model SS wymaga 2 godziny obróbki  na maszynie M$_1$, 1 godzina obróbki na maszynie M$_2$ i 3 godziny obróbki na maszynie M$_3$.  Do wyprodukowania modelu S wymagane są 2 godziny obróbki  na maszynie M$_1$, 3 godziny obróbki na maszynie M$_2$ i 1 godziny obróbki na maszynie M$_3$, a do modelu O 5 godzin obróbki  na maszynie M$_1$, 2 godziny obróbki na maszynie M$_2$ i 3 godziny obróbki na maszynie M$_3$. Z planów produkcyjnych wynika, że w ciągu tygodnia maszyny będą pracować przy produkcji akumulatorów nie dłużej niż: M$_1$ 40 godzin, pozostałe po 30 godzin. Wiedząc, że zyski jednostkowe wynoszą: z modelu SS - 32 zł, z modelu S - 24 zł i z modelu 48 zł, określić optymalną tygodniową produkcję akumulatorów, przy jakiej zysk przedsiębiorstwa będzie maksymalny. 

Należy:

\begin{enumerate}
\item Znaleźć postać dualną problemu
\item Określić wrażliwość rozwiązania optymalnego na zmiany cen akumulatorów.
\end{enumerate}
\end{zad}

\begin{zad}[5 pt.]
Asortyment zakładu produkującego meble szkolne stanowią ławki, stoły i krzesła drewniane. Produkcja każdego wyrobu odbywa się kolejno na trzech wydziałach produkcyjnych: na wydziale obróbki wstępnej drewna, w stolarni i wykańczalni, których dopuszczalny czas pracy (wynikający z liczby zatrudnionych pracowników) wynosi odpowiednio: 960, 800 oraz 320 godzin. W poniższej tablicy podano czas obróbki każdego wyboru na poszczególnych wydziałach oraz zyski uzyskiwane przez zakład ze sprzedaży wyrobów.

\begin{table}[htbp]
\begin{center}

\begin{tabularx}{\textwidth}{|Y|Y|Y|Y|Y|}
\hline
& \multicolumn{3}{|c|}{Nakład czasu pracy na jednostkę} &   \\
\multirow{2}{*}{Wyroby}&\multicolumn{3}{|c|}{wyrobu na wydziale}& Zysk\\ 
\cline{2-4}
&obróbki&\multirow{2}{*}{stolarni}&\multirow{2}{*}{wykańczalni}&jednostkowy\\
&wstępnej& & & \\
\hline
Ławka &8&8&4&60\\
Stół&6&4&3&30\\
Krzesło&1&3&1&20\\
\hline
\end{tabularx}
\end{center}
\end{table}

Należy:
\begin{enumerate}
	\item Znaleźć postać dualną problemu
	\item Określić wrażliwość rozwiązania optymalnego na zmiany cen mebli.
\end{enumerate}

\end{zad}



\end{document}
%%%%%%%%%%%%%%%%%%%%%%%%%%%%%%%%%%%%%%%%%%%%%%%%%%%%%%%%%%%%%%%%%%%%%%
